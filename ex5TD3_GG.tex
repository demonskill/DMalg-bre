\documentclass{article}
\renewcommand*\familydefault{\sfdefault}
\usepackage[utf8]{inputenc}
\usepackage[T1]{fontenc}
\setlength{\textwidth}{481pt}
\setlength{\textheight}{650pt}
\setlength{\headsep}{10pt}
\usepackage{amsfonts}
\usepackage[T1]{fontenc}
\usepackage{calrsfs}
\usepackage{geometry}
\geometry{ left=3cm, top=2cm, bottom=2cm, right=2cm}
\usepackage{xcolor}
\usepackage{amsmath}
\usepackage{stmaryrd}
\begin{document}
\title{Exercice 5 TD 3}
\author{Guillaume Gatt, Pierre Goutagny}
\maketitle
2) On pose $n=(dim(E))$ \\
Soit $(e_i)_{ 1\leq i \leq n}$ une base de E. \\
On pose les endomorphismes $e_{i,j}$ de la manière suivant : \\
$\forall (i,j,k) \in \llbracket 1,n\rrbracket^3,\ e_{i,j}(e_k)=\begin{cases}
e_j \ si \ i=k \\
0 \ sinon
\end{cases}$ \\
Montrons que $(e_{i,j})_{ 1\leq i,j \leq n}$ est une base de $Hom(E)$. \\
On a $|(e_{i,j})_{ 1\leq i,j \leq n}|=n^2=dim(Hom(E))$. \\
Montrons que cette famille est libre : \\
Soit $(\lambda_{i,j})_{1 \leq i,j \leq n}$ tel que $\sum_{i,j \in \llbracket 1 , n \rrbracket^2} \lambda_{i,j} e_{i,j} =0$ \\
On donc pour tout $k \in \llbracket1,n\rrbracket$ : \\
$\sum_{i,j \in \llbracket 1 , n \rrbracket^2} \lambda_{i,j} e_{i,j}(e_k) =0$ \\
$\sum_{j=1}^{n} \lambda_{k,j} e_j =0$ \\
On a donc comme $(e_i)_{ 1\leq i \leq n}$ est libre : \\
$\forall k \in \llbracket1,n\rrbracket \forall j \in \llbracket 1,n \rrbracket, \lambda_{k,j}=0$ \\
donc $(e_{i,j})_{ 1\leq i,j \leq n}$ est libre et a pour cardinal la dimension de Hom(E) donc c'est une base. \\ \\
Montrons que $\overline{\psi} \circ \varphi^{-1}(e_{i,j})=tr(e_{i,j})$ : \\
Montrons que $\varphi(e^\ast_i \otimes e_{j})= e_{i,j}$ où $(e^*_i)_{ 1\leq i\leq n}$ sont les formes linéaires coordonnées à la base $(e_i)_{ 1\leq i \leq n}$ \\
$\varphi(e^\ast_i \otimes e_{j})$  est l'application $z \rightarrow e^*_i(z)e_j$ \\
Il s'agit donc de l'application qui pour $k\in \llbracket1,n\rrbracket$ : \\
$\varphi(e^\ast_i \otimes e_{j})(e_k)=\begin{cases}
e_j \ si \ i=k \\
0 \ sinon \\
\end{cases}$ \\
donc $\varphi(e^\ast_i \otimes e_{j})= e_{i,j}$. \\
On a donc $\overline{\psi} \circ \varphi^{-1}(e_{i,j})= \overline{\psi}(e^\ast_i \otimes e_{j})=e^*_i(e_j)=\begin{cases}
1 \ si \ i=j \\
0 \ sinon \\
\end{cases}=tr(e_{i,j})$ \\
On a donc $\forall (i,j) \in \llbracket1,n\rrbracket^2,\overline{\psi} \circ \varphi^{-1}(e_{i,j})=tr(e_{i,j})$ \\
On a donc comme $(e_{i,j})_{ 1 \leq i,j \leq n}$ est une base on a pour $u \in Hom(E)$ : \\
$\overline{\psi} \circ \varphi^{-1}(u)=\overline{\psi} \circ \varphi^{-1}(\sum_{ 1 \leq i,j\leq n} \lambda_{i,j} e_{i,j})=\sum_{ 1 \leq i,j\leq n} \lambda_{i,j}\overline{\psi} \circ \varphi^{-1}(e_{i,j})=\sum_{ 1 \leq i,j\leq n}\lambda_{i,j} tr(e_{i,j})$ \\
$\overline{\psi} \circ \varphi^{-1}(u)= tr(\sum_{ 1 \leq i,j\leq n}\lambda_{i,j} e_{i,j})=tr(u)$ \\
Donc $\overline{\psi} \circ \varphi^{-1}=tr$ \\
\\
4) Cherchons $t\in E^* \otimes F$ tel que $rg(\varphi(t))\geq 2$. \\
Comme $dim(E)\geq 2$ il existe $e_1$ et $e_2$ deux vecteur non colinéaire de E, on complète cette famille libre $(e_1,e_2)$ en une base de E $(e_i)_{1\leq i \leq n}$.  \\
On considère $e^*_i \in E^*$ les formes linéaire coordonnées dans la base $e_i$ à cette base. \\
De même avec $m=dim(F) \geq 2$ on construit $(f_i)_{1\leq i \leq m}$ une base de F. \\
On pose $t=e^*_1 \otimes f_1 + e^*_2 \otimes f_2$ \\
on a donc $\varphi(t)$ est l'application $z \in E \rightarrow e^*_1(z) f_1 + e^*_2(z) f_2$. \\
On a donc $\varphi(t)(e_1)=f_1$ et $\varphi(t)(e_2)=f_2$ donc $(f_1,f_2) \in Im(\varphi(t))$ \\
Comme $Im(\varphi(t))$ est un espace vectoriel, $Vect(f_1,f_2) \subset Im(\varphi(t))$ \\
On a donc comme $f_1$ et $f_2$ ne sont pas colinéaire $l(t)=rg(\varphi(t)=dim(Im(\varphi(t))) \geq 2$ \\
donc t ne peut s'écrire comme un tenseur simple
\end{document}
