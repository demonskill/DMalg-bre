\documentclass{article}
\usepackage[T1]{fontenc}

\usepackage[margin=1.5cm]{geometry}
\usepackage[shortlabels]{enumitem}
\usepackage[fleqn]{amsmath}
\usepackage{amsfonts, mathrsfs}
\usepackage{amssymb}
\usepackage{commath, mathtools}
\usepackage{stmaryrd}
\DeclareMathOperator{\tr}{\mathrm{tr}}
\DeclareMathOperator{\rg}{\mathrm{rg}}
\def \r {{\ell(t)}}

\title{DM 2 :Exercice 5 TD 3}
\author{Guillaume Gatt \and Pierre Goutagny}
\date{20 Octobre 2020}

\begin{document}
\maketitle
\thispagestyle{empty}
\begin{enumerate}[1., start=1]
    \item \begin{align*}
            \psi \colon E^{\ast} \times E &\longrightarrow k \\
                   (\lambda, x) &\longmapsto \lambda(x)
          \end{align*}

    \item
    On pose $n=(dim(E))$ \\
   Soit $(e_i)_{ 1\leq i \leq n}$ une base de E. \\
   On pose les endomorphismes $e_{i,j}$ de la manière suivant : \\
   $\forall (i,j,k) \in \llbracket 1,n\rrbracket^3,\ e_{i,j}(e_k)=\begin{cases}
   e_j \ si \ i=k \\
   0 \ sinon
   \end{cases}$ \\
   Montrons que $(e_{i,j})_{ 1\leq i,j \leq n}$ est une base de $Hom(E)$. \\
   On a $|(e_{i,j})_{ 1\leq i,j \leq n}|=n^2=dim(Hom(E))$. \\
   Montrons que cette famille est libre : \\
   Soit $(\lambda_{i,j})_{1 \leq i,j \leq n}$ tel que $\sum_{i,j \in \llbracket 1 , n \rrbracket^2} \lambda_{i,j} e_{i,j} =0$ \\
   On donc pour tout $k \in \llbracket1,n\rrbracket$ : \\
   $\sum_{i,j \in \llbracket 1 , n \rrbracket^2} \lambda_{i,j} e_{i,j}(e_k) =0$ \\
   $\sum_{j=1}^{n} \lambda_{k,j} e_j =0$ \\
   On a donc comme $(e_i)_{ 1\leq i \leq n}$ est libre : \\
   $\forall k \in \llbracket1,n\rrbracket \forall j \in \llbracket 1,n \rrbracket, \lambda_{k,j}=0$ \\
   donc $(e_{i,j})_{ 1\leq i,j \leq n}$ est libre et a pour cardinal la dimension de Hom(E) donc c'est une base. \\ \\
   Montrons que $\overline{\psi} \circ \varphi^{-1}(e_{i,j})=tr(e_{i,j})$ : \\
   Montrons que $\varphi(e^\ast_i \otimes e_{j})= e_{i,j}$ où $(e^*_i)_{ 1\leq i\leq n}$ sont les formes linéaires coordonnées à la base $(e_i)_{ 1\leq i \leq n}$ \\
   $\varphi(e^\ast_i \otimes e_{j})$  est l'application $z \rightarrow e^*_i(z)e_j$ \\
   Il s'agit donc de l'application qui pour $k\in \llbracket1,n\rrbracket$ : \\
   $\varphi(e^\ast_i \otimes e_{j})(e_k)=\begin{cases}
   e_j \ si \ i=k \\
   0 \ sinon \\
   \end{cases}$ \\
   donc $\varphi(e^\ast_i \otimes e_{j})= e_{i,j}$. \\
   On a donc $\overline{\psi} \circ \varphi^{-1}(e_{i,j})= \overline{\psi}(e^\ast_i \otimes e_{j})=e^*_i(e_j)=\begin{cases}
   1 \ si \ i=j \\
   0 \ sinon \\
   \end{cases}=tr(e_{i,j})$ \\
   On a donc $\forall (i,j) \in \llbracket1,n\rrbracket^2,\overline{\psi} \circ \varphi^{-1}(e_{i,j})=tr(e_{i,j})$ \\
   On a donc comme $(e_{i,j})_{ 1 \leq i,j \leq n}$ est une base on a pour $u \in Hom(E)$ : \\
   $\overline{\psi} \circ \varphi^{-1}(u)=\overline{\psi} \circ \varphi^{-1}(\sum_{ 1 \leq i,j\leq n} \lambda_{i,j} e_{i,j})=\sum_{ 1 \leq i,j\leq n} \lambda_{i,j}\overline{\psi} \circ \varphi^{-1}(e_{i,j})=\sum_{ 1 \leq i,j\leq n}\lambda_{i,j} tr(e_{i,j})$ \\
   $\overline{\psi} \circ \varphi^{-1}(u)= tr(\sum_{ 1 \leq i,j\leq n}\lambda_{i,j} e_{i,j})=tr(u)$ \\
   Donc $\overline{\psi} \circ \varphi^{-1}=tr$
   \newpage
    \item \'Ecrivons $t$ comme somme de $\r$ tenseurs simples :\\
        $t = \sum_{i=1}^\r \lambda_i \otimes x_i$, avec $\lambda_i \in E^\ast$ et $x_i \in F$ pour $1 \leq i \leq \r$ \\
        Soit $z\in \ker\varphi(t)$.\\
        Alors $\sum_{i=1}^\r \lambda_i(z)x_i=0$\\
        Or les $(x_i)_{1\leq i \leq \r}$ sont libres.

        \par \setlength{\leftskip}{.2cm}
            Sinon, un des $e_i$ est une combinaison linéaire des autres. Supposons sans perdre de généralité que ce soit $x_1$. Alors il existe $\mu_2, \ldots, \mu_\r$ des scalaires tels que $x_1 = \sum_{i=2}^\r\mu_i x_i$.\\
            On a alors
            \begin{align*} t &= \sum_{i=1}^\r \lambda_i \otimes x_i \\
                &= \lambda_1\otimes\left(\sum_{i=2}^\r\mu_i x_i \right) + \sum_{i=2}^\r \lambda_i \otimes x_i \\
                &= \sum_{i=2}^\r (\lambda_i + \mu_i \lambda_1) \otimes x_i \text{\quad par bilinéarité de }\otimes
            \end{align*}
            Ainsi on peut écrire $t$ comme somme de $\r - 1$ vecteurs simples : contradiciton.

        \par \setlength{\leftskip}{0cm}
        Tous les $\lambda_i(z)$ sont donc nuls.\\
        Donc $\displaystyle\ker\varphi(t) \subseteq \bigcap_{i=1}^\r\ker\lambda_i$\\
        De plus on a l'inclusion réciproque :

        \par \setlength{\leftskip}{.2cm}
            En effet, si $z\in \bigcap_{i=1}^\r\ker\lambda_i$, on a
            \begin{align*}
                \varphi(t)(z) &= \sum_{i=1}^\r\lambda_i(z)x_i \\
                              &= \sum_{i=1}^\r 0 \\
                              &= 0
            \end{align*}

        \par \setlength{\leftskip}{0cm}
        On en conclut que $\displaystyle\ker \varphi(t) = \bigcap_{i=1}^\r\ker\lambda_i$, puis l'égalité des dimensions.\\
        Or les $\lambda_i$ sont libres, d'après le même argument que pour les $x_i$, donc $\displaystyle \dim\bigcap_{i=1}^\r\ker\lambda_i = n-\r$\\
        Ainsi, en appliquant le théorème du rang à $\varphi(t)$, on obtient
        \boxed{\rg\varphi(t) = \r}
    \item
    Cherchons $t\in E^* \otimes F$ tel que $rg(\varphi(t))\geq 2$. \\
    Comme $dim(E)\geq 2$ il existe $e_1$ et $e_2$ deux vecteur non colinéaire de E, on complète cette famille libre $(e_1,e_2)$ en une base de E $(e_i)_{1\leq i \leq n}$.  \\
    On considère $e^*_i \in E^*$ les formes linéaire coordonnées dans la base $(e_i)_{1 \leq i \leq n}$ à cette base. \\
    De même avec $m=dim(F) \geq 2$ on construit $(f_i)_{1\leq i \leq m}$ une base de F. \\
    On pose $t=e^*_1 \otimes f_1 + e^*_2 \otimes f_2$ \\
    On a donc $\varphi(t)$ est l'application $z \in E \rightarrow e^*_1(z) f_1 + e^*_2(z) f_2$. \\
    On a donc $\varphi(t)(e_1)=f_1$ et $\varphi(t)(e_2)=f_2$ donc $(f_1,f_2) \in Im(\varphi(t))^2$ \\
    Comme $Im(\varphi(t))$ est un espace vectoriel, $Vect(f_1,f_2) \subset Im(\varphi(t))$ \\
    On a donc comme $f_1$ et $f_2$ ne sont pas colinéaire $l(t)=rg(\varphi(t))=dim(Im(\varphi(t))) \geq 2$ \\
    donc t ne peut s'écrire comme un tenseur simple.

\end{enumerate}
\end{document}
