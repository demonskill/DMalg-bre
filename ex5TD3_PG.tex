\documentclass{article}
\usepackage[T1]{fontenc}

\usepackage[margin=1.5cm]{geometry}
\usepackage[shortlabels]{enumitem}
\usepackage[fleqn]{amsmath}
\usepackage{amsfonts, mathrsfs}
\usepackage{amssymb}
\usepackage{commath, mathtools}
\usepackage{stmaryrd}
\DeclareMathOperator{\tr}{\mathrm{tr}}
\DeclareMathOperator{\rg}{\mathrm{rg}}
\def \r {{\ell(t)}}

\title{Exercice 7}
\author{Guillaume Gatt \and Pierre Goutagny}
\date{20 Octobre 2020}

\begin{document}
\thispagestyle{empty}
\begin{enumerate}[1., start=1]
    \item \begin{align*}
            \psi \colon E^{\ast} \times E &\longrightarrow k \\
                   (\lambda, x) &\longmapsto \lambda(x)
          \end{align*}

    \item

    \item \'Ecrivons $t$ comme somme de $\r$ tenseurs simples :\\
        $t = \sum_{i=1}^\r \lambda_i \otimes x_i$, avec $\lambda_i \in E^\ast$ et $x_i \in E$ pour $1 \leq i \leq \r$ \\
        Soit $z\in \ker\varphi(t)$.\\
        Alors $\sum_{i=1}^\r \lambda_i(z)x_i=0$\\
        Or les $(x_i)_{1\leq i \leq \r}$ sont libres.

        \par \setlength{\leftskip}{.2cm}
            Sinon, un des $e_i$ est une combinaison linéaire des autres. Supposons sans perdre de généralité que ce soit $x_1$. Alors il existe $\mu_2, \ldots, \mu_\r$ des scalaires tels que $x_1 = \sum_{i=2}^\r\mu_i x_i$.\\
            On a alors 
            \begin{align*} t &= \sum_{i=1}^\r \lambda_i \otimes x_i \\
                &= \lambda_1\otimes\left(\sum_{i=2}^\r\mu_i x_i \right) + \sum_{i=2}^\r \lambda_i \otimes x_i \\
                &= \sum_{i=2}^\r (\lambda_i + \mu_i \lambda_1) \otimes x_i \text{\quad par bilinéarité de }\otimes
            \end{align*}
            Ainsi on peut écrire $t$ comme somme de $\r - 1$ vecteurs simples : contradiciton.

        \par \setlength{\leftskip}{0cm}
        Tous les $\lambda_i(z)$ sont donc nuls.\\
        Donc $\displaystyle\ker\varphi(t) \subseteq \bigcap_{i=1}^\r\ker\lambda_i$\\
        De plus on a l'inclusion réciproque :

        \par \setlength{\leftskip}{.2cm}
            En effet, si $z\in \bigcap_{i=1}^\r\ker\lambda_i$, on a
            \begin{align*}
                \varphi(t)(z) &= \sum_{i=1}^\r\lambda_i(z)x_i \\
                              &= \sum_{i=1}^\r 0 \\
                              &= 0
            \end{align*}

        \par \setlength{\leftskip}{0cm}
        On en conclut que $\displaystyle\ker \varphi(t) = \bigcap_{i=1}^\r\ker\lambda_i$, puis l'égalité des dimensions.\\
        Or les $\lambda_i$ sont libres, d'après le même argument que pour les $x_i$, donc $\displaystyle \dim\bigcap_{i=1}^\r\ker\lambda_i = n-\r$\\
        Ainsi, en appliquant le théorème du rang à $\varphi(t)$, on obtient
        \boxed{\rg\varphi(t) = \r}

            
\end{enumerate}
\end{document}

